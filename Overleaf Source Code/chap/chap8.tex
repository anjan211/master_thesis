\chapter{Challenges and limitations} \label{challenges}

\bigskip \bigskip 

While there are many potential benefits to using a knowledge graph for industrial risk assessment, there are also several challenges and limitations that should be considered. These challenges are discussed in this chapter. The challenges can primarily segregated into two sections.
\begin{enumerate}
    \item Challenges faced in the development process
    \item Potential challenges in the implementation
\end{enumerate}

\section{Challenges faced in the development process}

In this section, the challenges that came across the development process of the knowledge graph is discussed. The part of development which requires further research and development is the section for input data \ref{gather_relevant_data}. The quality and completeness of the data used to populate the knowledge graph are critical factors that can impact the effectiveness of risk assessment. If the data is incomplete, inconsistent, or inaccurate, it can lead to incorrect assessments. An improper risk assessment can be dangerous. Therefore, it is essential to establish an assurance process of the quality of input data to ensure that the data used to populate the knowledge graph is accurate, consistent, and complete and avoid GIGO (garbage in, garbage out) as much as possible. However, it must be done in the most automated way possible due to the huge volume of text data as input. 

\paragraph{} The development of an ontology that accurately represents the concepts and relationships from the safety standard should be made efficient and accurate. According to F. N. Al-Aswadi et al \cite{al-aswadi_chan_gan_2019} fully automatic construction of ontologies from text is still a major challenge. The semi-automated approach used to generate SPO triples from text for this thesis work is also manual labour intensive. Most of the effort required towards development of the knowledge graph went towards perfecting the SPO triples manually. Therefore, further development of ontology creation techniques that require less human intervention still remains a challenge and is an interesting area to study.

\paragraph{} Safety standards are sometimes not well defined. Developing safety standards requires a general agreement among stakeholders, which can be difficult to achieve. Different stakeholders may have different priorities, perspectives, and interests, which can make it challenging to develop standards that are widely accepted and effective. Safety standards thus may vary across different cultures and regions, reflecting differences in values, and priorities. This can make it challenging to develop a knowledge graph which is consistent, uniform and accurate for the safety standards. It requires general understanding of the standards and the new risks and hazards to develop a knowledge graph which could again be a big challenge towards automatic development of knowledge graphs from standards.

\paragraph{} If the knowledge graph is built on a large scale for a TIC organisation, there would be massive amount of data to build the model. It means digitisation of a large number of safety standards and test documents. Construction and maintenance of such quantity of data requires a dedicated department which could be expensive. It requires investment in data input, ontology development, computing resources, and software development. Dealing with large and complex datasets may require investment in high-performance computing infrastructure and specialized software tools to support the development and maintenance of the knowledge graph. Also updating the knowledge graph is significantly important as risks are constantly evolving. As technology advances, new safety issues may arise that are not covered by existing standards. This can create gaps in risk assessments, which can leave unmitigated risks. This makes it an important as well as a challenging task to update the knowledge graph efficiently. So developing the Machine-to-Machine communication of risks \ref{m2m}, which is a challenging task in itself is also required to be worked on simultaneously.

\section{Potential challenges in the implementation}
In this section, the challenges that can hinder the implementation of a knowledge model in industrial risk assessments is discussed. There are several reasons that can keep it far from achieving industrial risk assessments using knowledge models. The success depends heavily on user acceptance and adoption which is also the main challenge to overcome. Users here are the person who would perform the risk assessment using the new technology.

\paragraph{} Users may be resistant to change and may prefer to use existing tools and processes. This can be particularly challenging when introducing a new technology like a knowledge graph, which may require changes to existing workflows and processes. Users may not be aware of the benefits of a knowledge graph or how it can be used to support decision-making. This can impact user acceptance, as users may not see the value in using a knowledge graph. To address this challenge, it is important to communicate the benefits of using a knowledge graph and to engage users in the development and implementation of the technology. Providing user training and support can also help to raise awareness of the benefits of using a knowledge graph and can also help to ease the transition to the new technology.

\paragraph{} Knowledge graphs can be complex, particularly when dealing with large and diverse datasets. This complexity can make it challenging for users to navigate and understand the information presented in the knowledge graph, which can impact user acceptance. Knowledge graphs sometimes may require technical knowledge to use effectively. Users may need to understand the underlying ontology and the relationships between concepts in the domain. This can be a barrier to adoption for users who do not have the required technical knowledge or training. To address this challenge, it is important to design an intuitive user interface with which the user can interact easily and can get the desired information quickly, without requiring a deep understanding of the underlying technology. 

\paragraph{} Finally, organizational stand point is the most important aspect to embrace a change. The culture and organizational structure of an organization can impact user acceptance directly. Some organizations may be more open to adopting new technologies, while others may be more resistant to change. Orthodox organizations may fear actual implementation of new technologies especially if they are disruptive to existing processes or require significant investment. This can pose as a major challenge. To address this challenge, it may be necessary to build a culture of innovation and openness to change within the organization. This can involve engaging with stakeholders at all levels of the organization, communicating the benefits of using a new technology which can make risk assessments more efficient. It is required to start small with the implementation and test their effectiveness and impact on existing workflows. A change can bring a competitive advantage to stay ahead of the curve and develop products and services that meet the needs of a changing world. 