\chapter{Conclusion} \label{conclusion}

\bigskip \bigskip The thesis is concluded by discussing the main contributions and an overall summary of the work done. The contribution contains the results and conclusions for each research question that were discussed in \ref{rq}.

\section{Contributions}

The thesis covers the following research questions.

\rqformat{\textbf{RQ 1:} How can industrial safety assessments be made faster and efficient?}

\paragraph{} This is a broad question and can further be divided into the further questions which follows. But as a general answer, industrial safety assessment can be made faster and efficient by changing the current workflow. Currently, risk assessment is mostly done manually and is primarily based on an expert's interpretation of safety standards. Also safety standards are not always well-defined and the requirements are sometimes vague. This makes the current process of risk assessment slow and inefficient if it is done manually and only at an expert's discretion. However, the process can be made faster and efficient by digitizing the workflow of current risk assessment. This thesis, mainly focuses on the digitization of knowledge captured from the experts and from safety standards. 

\rqformat{\textbf{RQ 2:} How can knowledge from experts and standards be captured for re-use?}

\paragraph{} To address this question, a knowledge model is developed for industrial risk assessment. The complete methodology is discussed in Chapter \ref{method}. The two sources of input data of this model are the safety standard ISO 12100 and an expert's knowledge from a previous risk assessment. A semi-automated approach is adopted to extract knowledge which is described in \ref{gather_relevant_data}. An NLP based information extraction pipeline is used to extract knowledge in a structured way from general text data. But this algorithm is not always accurate, so manual work is still required for knowledge extraction. Knowledge is extracted as SPO (Subject-Predicate-Object) triples. Semantic technology like a knowledge graph is considered to use for this work since the input data for the model is textual and can best fit the purpose. A class hierarchy structure as defined in \ref{class_hierarchy} is prepared as an outline, which helps to build the knowledge graph. The tools used to develop the knowledge graph are described in \ref{develop} along with the development process in which the SPO triples are structured into a knowledge graph. Most of the process is automated wherever possible. 

\rqformat{\textbf{RQ 3:} How to deal with the missing information regarding the performance level indicators?} 

\paragraph{} This question can be answered once the knowledge model is prepared. The solution can be found with the implementation of the knowledge model in Chapter \ref{implementation}, section \ref{specific_use}. A simple query to the model for performance level can show the minimum performance level required by a safety component to mitigate the risk. This is beneficial in risk assessment as the expert now no longer need to search the safety standard to find the required performance level, which can sometimes be difficult to find. Apart from this, Chapter \ref{implementation} also contains other use cases where the knowledge graph can benefit the current risk assessment process.

\rqformat{\textbf{RQ 3:} How can the knowledge model be validated?}

Validation is the most important part of developing the knowledge model since it is built to be used in industrial safety assessments. So in Chapter \ref{validation}, it is discussed how to ensure that the model have a high face validity and a good quality output which is also a validated by a human expert. Face validity is performed by going through the model thoroughly along with the safety standard by the process described in \ref{face_validity} and fix errors simultaneously. This serve as a first step of sanity check for the model correctness. Expert validation of the output from the model is the next important step for validation to ensure its accuracy and reliability. This is done by generating a checklist for few clauses from the knowledge graph and presenting the output in a familiar format of a TRF (Test Report Form) to the expert. This makes it easy to go through the output for the expert and it is also the fastest way of validating the knowledge graph. An evaluation sheet is also prepared with which the expert can give a quantifiable result of the knowledge model.

\section{Summary}

In conclusion, this thesis proposes the development of a knowledge model to support industrial risk assessment. The introduction begins with the motivation to work for this thesis from the perspective of a TIC (Testing, Inspection and Certification) industry. The problems that can come up while working for this thesis are also discussed along with the research questions and proposed hypotheses which are addressed in the section above. Then the background of how industrial risk assessment is performed currently is presented, which contains a brief about safety standards and the ISO 12100. The different techniques to model knowledge and how modeling of knowledge can help industrial risk assessment is also discussed. A literature review is also conducted on the works that are related to this thesis.

\paragraph{} The main part of the work contains a systematic methodology to develop the knowledge model. The scope and purpose of the model is discussed. Extracting knowledge from the safety standard and the expert is an important task which needs to be done most efficiently. Here, a semi-automated approach is discussed which is used for knowledge extraction. The concepts and relations are structured into SPO (Subject-Predicate-Object) triples by this approach. Then the knowledge graph is prepared based on a hierarchical structure of the concepts. The SPO triples is formalized with ObjectLogic and the knowledge graph is finally built with OntoBroker. After development, it is also necessary to implement the model in several use cases of risk assessment. A few implementations are discussed which can be done with the model built. Validation of the model is the most important part of building a model and it is done against an expert. A quantitative review of the output from the model ensures the accuracy and reliability of the model.

\paragraph{} After the development phase, future applications of the model is discussed. How the model can be improved and what further developments can be made for future risk assessments. An interesting concept of machine-to-machine communication is also discussed for a futuristic automated risk assessment. Finally, the challenges faced in the development process are discussed along with the challenges that can hinder the implementation of knowledge model in risk assessments.