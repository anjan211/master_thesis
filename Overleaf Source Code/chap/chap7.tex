\chapter{Future Applications} \label{applications}

\bigskip \bigskip 

The main benefit of this work is towards making a faster, cheaper and safer industrial risk assessment.   
\begin{itemize}
	\item \textbf{Faster}: The risk assessments can be reused and hence machines can be tested faster.
	\item \textbf{Cheaper}: Less time required for assessment and less manual work required means lower costs involved.
	\item \textbf{Safer}: Safe assessment only with the relevant requirements and information per project. This can help to identify hazard and mitigate them easily.
\end{itemize}

With knowledge being digitized, the expert's interpretation of a standard can be structured into a knowledge model. This in turn would help in future risk assessments. The fuzzy requirements from standards can be interpreted correctly as the data from the standard now is not only just text but converted into a knowledge graph with nodes and edges. The structured knowledge from domain expert can help in refining the requirements better. With the information being captured by a knowledge model and is described in a machine-understandable way, it paves a path for an artificially intelligent (“smart”) software. The smart software could be able to reason about this model. This can incur new facts (or requirements) to be derived automatically from the knowledge model, or existing facts (or requirements) can be verified. 

\paragraph{} With a semantic model in place, machine to machine communication of risk assessment for risk assessment can be made possible. New hazards can be identified and can be mitigated automatically. 

\section{Implementation in the tool for risk assessment}

Apart from the implementations and uses as mentioned in chapter \ref{implementation}, the knowledge graph can also be used in the current risk assessment process in other ways. To harness the full capabilities of a knowledge graph, it can be integrated with a risk assessment software just like mCom ONE. This can be done by developing an integration layer that allows the software application to access and query the knowledge graph. This may involve developing APIs, data connectors, or other integration mechanisms. This implementation is not done in this thesis but can be considered as an extension to this project into an already existing tool. This integration can be beneficial in the following ways:

\begin{enumerate}
    \item Data Integration: The knowledge graph is used to integrate data from different sources and formats into a single coherent model. This can be useful when the application mCom ONE needs to access data from different sources, such as a combination of structured and unstructured data, and across different domains. By integrating data of a knowledge graph, the application can have a more complete understanding of the data, which can enable more sophisticated analysis, risk assessment and decision-making. 
    \item Semantic Search: A knowledge graph can be used to improve the accuracy and relevance of search results from the application. The knowledge graph can enable more precise and relevant search results, by understanding the relationships between entities and concepts. This can help users quickly find relevant information and insights, which can support more effective risk management.
    \item Personalized Recommendations: The knowledge graph can enable more accurate and personalized recommendations, based on the relationships between entities and concepts. This can support more effective risk management, by tailoring recommendations to the specific needs and requirements of individual users. For example, while evaluating a particular hazard, similar related hazard that can also co-exist can pop-up as recommendation to be considered alongside.
    \item Improved Collaboration and Communication: The knowledge graph can enable more effective collaboration and communication between different stakeholders involved in risk management. By providing a shared understanding of the data and its relationships, the knowledge graph can support more efficient and effective decision-making processes.
\end{enumerate}

Looking at the above benefits, existing features in mCom ONE can be improved as well as new features can be added. The help option can be improved. A chat-bot can be developed to give quick information about the safety standard. A recommender system can be developed for the search bar which can give related results when searching for a hazard. The hazard pool can be automatically updated with data from the knowledge graph. Hazards and control measures can be better connected. The checklist based audit can be improved where safety standards are not always well defined and requirements which are sometimes fuzzy. With semantic technologies, like the knowledge graph in place, software applications for industrial risk assessment like mCom ONE can be made better. Since, a knowledge graph have the potential to significantly improve the accuracy, efficiency, and effectiveness of software systems and applications, and enable more advanced data processing, analysis, and decision-making capabilities \cite{text2kg}.

\section{Future development} \label{future_use}
The future applications of a knowledge model in industrial risk assessments are vast and varied, as technological advances and data-driven approaches continue to shape the field of industrial risk management. The knowledge model can be integrated with various other technologies. Here are some potential future applications:
\begin{itemize}
    \item \textbf{Integration with IoT:} The integration of knowledge models with Internet of Things (IoT) devices can enable real-time risk monitoring, as well as automated risk assessment and management. IoT devices such as sensors, cameras, and wearables can collect data on safety conditions, equipment status, and worker behavior. This data can be fed into the knowledge model, which can then analyze the input to identify potential safety hazards and suggest risk mitigation strategies. This can also happen in real-time with an actuator in place to mitigate the risk or an alarm to alert of a possible threat. For example, in a machine if a sensor (for example a camera) detects that the measures for the reduction of radiation at source are not adequate and the machine is not provided with adequate protective measures, such as use of guards or filtering according to 6.3.4.5 of ISO 12100, then an alarm for non-compliance can be put up in real-time and the risk of radiation emission can be mitigated. Thus integration of IoT devices is actually a head-start towards machine-to-machine communication in risk assessment.
    \item \textbf{Use of Artificial Intelligence:} The use of machine learning and artificial intelligence algorithms in knowledge models can improve risk assessment accuracy, identify complex patterns and relationships, and enable predictive maintenance and optimization. AI algorithms can help the model to understand human behaviour and patterns with better accuracy. The incorporation of human factors into knowledge models can enable a more comprehensive understanding of risks, as well as the development of human-centric risk management strategies that address the interaction between humans and machines. For example, with an AI algorithm in place which is being fed on several recent accident reports, it can detect a pattern of the most common risk that is causing accidents and the knowledge model can be queried for the risk and the corresponding preventive measures which can then be checked in the first place for future risk assessments. Thus such integration of AI algorithms with the knowledge model can suggest for the root causes of recent incidents and corrective actions to prevent similar incidents in the future. 
    \item \textbf{Simulation for risk assessment:} Simulation can be an useful addition to the knowledge model for risk assessment of complex systems. According to Zio \cite{ZIO2018176}, simulation can be a fundamental tool for knowledge retrieval. By running a set of simulations with different initial configurations of system design and operation parameters, the different system state can be evaluated along with the corresponding safety conditions. These safe states can then be recorded into the knowledge model. Thus, simulation can be a probable source for knowledge elicitation apart from safety standards and previous risk assessments. And the knowledge model can be enriched with a wide range of system information and safe conditions for several configurations.
\end{itemize}

Integrating these technologies with the knowledge model for risk assessments can provide more comprehensive and effective safety solutions, thereby, reducing the risk of incidents and improving safety culture in industrial settings. These improvements can enable a more efficient, and effective approach to industrial risk management, leading to better safety, reliability, and productivity in industrial settings. 

\section{Machine-to-machine communication} \label{m2m}
Machine-to-machine communication (M2M) of a knowledge model in industrial risk assessments can enable real-time and automated risk management processes as discussed by Bunte \cite{Bunte2020}. A similar work is done by Salim Khan et al. \cite{Khan2022}, the paper proposes an IoT-based system that uses Raspberry Pi and sensors to collect real-time data from machines, such as voltage, current, gas value and temperature. The data is then processed by a machine learning model that compares it with training data and generates statistical graphs of machine performance. The paper claims that the proposed system can help to monitor and analyze the machines in an industrial environment and predict the possibility of upcoming risks. 

\subsection{Advantages} 
Machine-to-machine communication of a knowledge model for industrial risk assessment has several advantages. One of the main benefits is that it enables more efficient and effective risk assessment processes. By automating the collection and analysis of safety data, M2M communication can reduce the time and effort required for risk assessment and help identify potential hazards more quickly and accurately. Additionally, M2M communication can enable real-time monitoring of safety conditions, allowing for rapid responses to emerging risks. This can be particularly important in dynamic environments such as manufacturing facilities, where conditions can change quickly and unpredictably.

\paragraph{} Another advantage of M2M communication is that it can help ensure that updated safety standards are consistently applied across an organization. By providing a shared knowledge model of safety standards that can be accessed and utilized by all relevant parties, M2M communication can help minimize inconsistencies in safety assessments and ensure that safety protocols are followed consistently across different departments and locations. This can be especially beneficial for large organizations with multiple locations and teams that may have different interpretations of safety standards. By having a shared knowledge model, which is being updated in real-time, all parties involved in industrial risk assessment can work from the same set of guidelines, reducing the risk of errors or oversights due to misunderstandings or misinterpretations of safety standards.

\paragraph{} By facilitating real-time data exchange between machines and systems, M2M communication can enable the collection and analysis of large amounts of data related to machine performance and maintenance needs. This data can then be used to identify patterns and trends, and to predict when machines are likely to require maintenance or replacement. With this predictive maintenance approach, potential machine failures can be detected and addressed, before they result in downtime or hazards. This helps to reduce the likelihood of accidents and injuries in the workplace, and can also result in significant cost savings by reducing the need for unplanned maintenance or equipment replacement. Overall, the use of M2M communication for predictive maintenance can improve both safety and efficiency in industrial settings.

\paragraph{} As new risks are identified, they can be added to the knowledge model and shared with relevant parties and machines in real-time, allowing for a more comprehensive and up-to-date understanding of potential risks. This can also enable more proactive risk management by allowing for the identification and mitigation of potential risks before they become major issues. Additionally, by using a shared semantic knowledge model, all machines can have a standardized and consistent understanding of the risks, leading to better communication and collaboration in automated risk management.

\paragraph{} The above potential future applications of the semantic model capable of machine-to-machine communication for risk assessment can easily be understood with a simple real life scenario. For example, if a machine working for over a decade has an accident, the hazard can be shared with similar machines that are prone to such accident. Also, the manufacturer can change some requirements to mitigate the hazard over the semantic web. By sharing the updated risks and hazards in a shared semantic knowledge model through M2M communication, other similar machines or equipment can benefit from the knowledge and take necessary precautions to prevent accidents. This can help to ensure that the hazard is not repeated in the future and that preventive measures are taken to mitigate the risk. Additionally, manufacturers can use this information to improve their designs and prevent similar accidents from happening in the future. The relevant safety standards can also be updated accordingly. Ultimately, this can save lives, reduce injuries, and minimize damage to property and equipment. The work presented in this master thesis would serve as the preliminaries of such use cases and this work can be extended towards machine communication of hazards.

\subsection{Limitations}
While M2M communication has many advantages and probable use cases, it has has some limitations which can be worked on. One limitation is that it requires a reliable and secure network to function properly. If the network infrastructure is not reliable, there may be issues with data transmission, which can result in delays, errors, or loss of data. This can impact the effectiveness of risk assessment processes. Another limitation for M2M communication can be the compatibility of the different systems used. If the systems are not compatible, it can be challenging to establish M2M communication between them, and this can hinder the effectiveness of risk assessments.

Also as identified by several works like in \cite{Kadena2017} and \cite{TUNA2017142} security risks are a major threat for M2M communication. Protection against unauthorized access and modification of safety parameters in the knowledge model must be prevented. All data exchanged between machines should be encrypted to prevent unauthorized interception. This can be achieved using various encryption protocols, such as Transport Layer Security (TLS) or Secure Sockets Layer (SSL). To detect and prevent potential security breaches, intrusion detection systems can be implemented. These systems can monitor network traffic and alert administrators of any suspicious activity or attempts at unauthorized access. A knowledge model for safety standards on cyber-physical systems can be employed here for continuous monitoring. IEC 62443 can be used according to the work of Halenar et al. \cite{Halenar2023} which proposes an architecture of the cybernetic systems in a smart
factory with a focus on communication security. Securing such cyber-physical system and M2M communications against cyber threats especially for safety projects of TIC industry can be an interesting and important topic for future research.

\paragraph{} M2M communication may require a significant investment in terms of infrastructure, and training. Organizations may need to invest in new technologies and equipment, and need to train their staff on how to use these new systems effectively. The initial investment and ongoing maintenance costs can be significant, and organizations need to weigh these costs against the potential benefits of M2M communication for risk assessment.