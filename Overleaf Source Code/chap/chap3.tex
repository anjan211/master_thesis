\chapter{Related Work} \label{rw}

\bigskip \bigskip To the best knowledge, there exists no prior work which encompasses all research topics together, as addressed in Chapter \ref{rq}. Therefore the related works are discussed topic wise.

\section{Knowledge elicitation}
Acquisition of knowledge and structuring it is itself a form of complex expertise. There are a wide range of techniques and methods that could be used to elicit knowledge from domain experts. The systematic methods to do so are illustrated in the work of Shadbolt et al. \cite{shadbolt2015}. The paper \cite{shadbolt2015} also presents a study on the software tools for knowledge acquisition. The paper focuses on knowledge elicitation in the context of ontology development, from domain experts. 

\paragraph{} Another article by Fenoglio et al. \cite{Fenoglio2022} describes a proposal for tacit knowledge elicitation to capture the best operational practices of experienced domain experts. This paper is based on a mix of algorithmic techniques and cooperation. It provides a functional description of a cognitive framework for capturing knowledge into a knowledge graph.

\paragraph{} The work by N. Jain et al. \cite{text2kg} presents an overall complete work package for automated construction of knowledge graph from raw text. It shows how OpenIE can be used to extract semantic triples directly from text, without the requirement of a formal ontology. This can be beneficial when building a knowledge model from scratch with unstructured text such as documented text as input. Therefore, these works can be referred to work on RQ 2 as given in Chapter \ref{rq}.

\section{Modeling}

The methodologies for the development of knowledge based systems are discussed in the work of Plant et al. \cite{Plant2003}. The "Buchanan's methodology" as cited here is an approach of interest which consists of 5 main phases namely Identification, Conceptualization, Formalization, Implementation and Testing. The problem is first identified, concepts are found to represent knowledge, the concepts are then structured in a formal arrangement, the knowledge is then implemented and tested on use cases to validate the system. The Reformulations, Redesigns and Refinements are based on the Testing phase and accordingly each phase is repeated. 

\paragraph{} There are works which deals with the modeling of knowledge from safety standards. The work presented in \cite{Luo2016} is aimed at model based safety assurance. The novel approach called the "Snowball methodology" described in this PhD thesis provides a rule based technique for developing conceptual model from safety standard. The technique is similar to creating a snowman, starting with very basic concepts that comes from high-level requirements and then just like rolling the snowball in the snow, the size of the model becomes bigger with more concepts and relations adding up to the model.

\paragraph{} In the work presented in \cite{Bunte2020}, an interoperability of semantic models in modular production systems is given. An ontology based approach is used to create a semantic model for individual modules, combined into an overall model. The data can be transferred to individual modules through the semantic model, which would play as an interface. This improves adaptability without manual effort. The use case, as presented in this PhD work, would be beneficial for faster re-assessments of product in safety and security assessments. If this similar use case is considered for the communication of information related to performance level from the semantic model to the actual machine, an M2M communication is achieved. This can be a two step approach, first find the missing information regarding performance level and then an automatic communication to the connected machines. Then if risk persists it can be mitigated automatically if possible, otherwise an alarm might be set. This can be an ultimate goal to reach towards automated risk assessment and can be a topic of future research. However, for this thesis, only finding the missing information related to performance level is focused on and the M2M communication is left for future work. Hence, \cite{Bunte2020} is useful for RQ 3 given in Chapter \ref{rq}.

\section{Validation}

According to Luo \cite{Luo2016}, validation by domain expert is a necessary final step of model development. A formal representation of the conceptual model is an excellent
basis for validation. Generating checklist from the given knowledge graph is a desirable achievement for faster risk assessment. This can also serve the purpose of validation. Paper \cite{9297991} introduces an ontology based approach for automating the generation of questions from an RDF graph. Information from the ontology is broken down into categories in the format of SPARQL queries. The queries are then converted into questions by the algorithm described in the paper.

\paragraph{} The work of R. Roy et al. \cite{Roy2004} also involves domain experts in the process of model validation. The work also presented a validation questionnaire for the experts to record their impression of the model and present a quantitative outlook of the model.